% !TeX root = ../main.tex

\chapter{算法实现与仿真}

模板使用 \pkg{natbib} 宏包来设置参考文献引用的格式,
更多引用方法可以参考该宏包的使用说明。
\section{基于 Rényi 熵异常初检模块实现与分析}
\section{基于深度神经网络 DDoS 流量检测模块实现与评估}
\custsetup{
  cite-style = super,
}
\noindent
\begin{tabular}{l@{\quad$\Rightarrow$\quad}l}
  \verb|\cite{knuth86a}|         & \cite{knuth86a}         \\
  \verb|\citet{knuth86a}|        & \citet{knuth86a}        \\
  \verb|\cite[42]{knuth86a}|     & \cite[42]{knuth86a}     \\
  \verb|\cite{knuth86a,tlc2}|    & \cite{knuth86a,tlc2}    \\
  \verb|\cite{knuth86a,knuth84}| & \cite{knuth86a,knuth84} \\
\end{tabular}


\section{仿真环境下的实验结果与性能评估}

\custsetup{
  cite-style = inline,
}
\noindent
\begin{tabular}{l@{\quad$\Rightarrow$\quad}l}
  \verb|\cite{knuth86a}|         & \cite{knuth86a}         \\
  \verb|\citet{knuth86a}|        & \citet{knuth86a}        \\
  \verb|\cite[42]{knuth86a}|     & \cite[42]{knuth86a}     \\
  \verb|\cite{knuth86a,tlc2}|    & \cite{knuth86a,tlc2}    \\
  \verb|\cite{knuth86a,knuth84}| & \cite{knuth86a,knuth84} \\
\end{tabular}



\section{本章小结}

\custsetup{
  cite-style = authoryear,
}
\noindent
\begin{tabular}{l@{\quad$\Rightarrow$\quad}l}
  \verb|\cite{knuth86a}|         & \cite{knuth86a}         \\
  \verb|\citep{knuth86a}|        & \citep{knuth86a}        \\
  \verb|\citet[42]{knuth86a}|    & \citet[42]{knuth86a}    \\
  \verb|\citep[42]{knuth86a}|    & \citep[42]{knuth86a}    \\
  \verb|\cite{knuth86a,tlc2}|    & \cite{knuth86a,tlc2}    \\
  \verb|\cite{knuth86a,knuth84}| & \cite{knuth86a,knuth84} \\
\end{tabular}

\custsetup{
  cite-style = super,
}

% 注意,参考文献列表中的每条文献在正文中都要被引用。这里只是为了示例。
\nocite{*}
