% !TeX root = ../main.tex

\chapter{绪论}%
%\section{研究背景、目标、意义}
\section{研究背景}
随着互联网的迅速发展和用户数量的激增,网络设备的多样性与数量不断增加,同时网络服务需求也在不断提高,传统的网络架构面临着各种挑战。传统的 IP 网络结构复杂、管理困难,控制域和数据域之间高度耦合,这些因素使得传统 IP 网络难以有效管理。

为解决上述问题,2009 年 McKeown 教授提出了 SDN 网络(Software Defined Networking)架构。SDN 作为网络虚拟化的一种实现方式,通过解耦网络的转发逻辑与控制逻辑,将整个网络架构划分为应用平面、控制平面和数据平面三个部分。其中,数据平面负责转发数据报文并执行控制平面下发的转发策略;控制平面则负责管理数据平面,是整个 SDN 网络的核心控制节点,其开放的多个编程接口有助于网络管理人员实现集中管理;应用平面则利用控制平面提供的接口扩展网络功能,满足上层业务需求。在 SDN 网络中,当新报文到达时,数据平面将其转发给控制平面进行路由决策;控制平面根据网络状况生成转发策略并下发到数据平面;而数据平面中的 OpenFlow 交换机具有智能化特性,能够动态适应网络变化,提高网络决策敏捷性,从而在自适应路由、拥塞控制、时延分析等方面具备良好性能。

分布式拒绝服务攻击(Distributed Denial of Service,DDoS)以低成本、广泛破坏等特点著称,其主要原理是利用大量僵尸主机对目标主机发起攻击,耗尽目标主机的资源,如 CPU 计算资源、TCP 连接、网络带宽等。2000 年 2 月,知名网站 CNN.com 曾遭受 DDoS 攻击导致瘫痪并造成巨大经济损失,自此,DDoS 攻击备受关注:2013 年 8 月 5 日,中国 .CN 服务器遭受大规模 DDoS 攻击,导致政府网站和服务无法访问;2015 年,英国广播公司(BBC)和 iPlayer 因 DDoS 攻击被迫下线。根据国家互联网应急服务中心发布的《2019 年中国互联网网络安全报告》,我国平均每天发生 1802 起攻击强度超过 1 Gibits 的 DDoS 攻击事件,较 2012 年增长 76%,攻击方式也日趋多样化。

传统网络设备的控制与转发紧密结合,由同一设备实现,而软件定义网络(SDN)则将控制平面与数据平面分离,成为新型网络架构的代表。SDN 的核心结构包括数据平面、控制平面和应用平面三层。其中,数据平面负责数据转发的交换机;控制平面包含一个或多个 SDN 控制器,集中管理网络的转发设备;应用平面主要服务于业务应用。在交换机和控制器之间使用开放的接口协议,如 OpenFlow 协议,以解决转发设备与控制器之间的通信问题。

OpenFlow 交换机包含流表,由匹配规则、计数器和动作字段组成,通过控制信令完成数据转发。SDN 由于架构特性存在着一些安全隐患,如劫持、中毒、配置错误、拒绝服务和跳板攻击等。特别是跳板攻击因易于执行且难以被发现,备受黑客青睐,黑客可以通过抓包工具如 wireshark 监听网络流量,轻松捕获控制器与 OpenFlow 交换机之间的控制信令。
\section{国内外研究现状}
\subsection{DDoS 攻击研究现状}
针对 DDoS 攻击,在国内,王庆生等人提出了基于注意力机制的 DDoS 攻击检测方法。该方法将根据领域知识提取的特征向量与数据预处理后的数据流矩阵结合,构建基于注意力机制的双向长短期记忆网络数据输入格式,显著提升了 DDoS 攻击的检测效果。

罗文华等基于 Hadoop 集群,利用 Map Reduce 模型的多属性融合检测算法,重新设计了分布式入侵检测架构,能够融合多个属性位并自适应调整阈值,在确定具体攻击类型的前提下,提高了攻击检测率。余学山等人利用智能蜂群算法和融合聚类方法进行 DDoS 攻击检测,所提出的算法不仅缩短了检测时间,还提高了检测的准确性。

蔡佳晔等人基于 Sibson 距离实现了 SDN 环境下的入侵检测。其分层式攻击架构在一定程度上缓解了控制器的压力,并提升了网络的稳定性。

Ava Tahmasebi 等人采用排队论来评估流量到达,提出了一种新的 SDN 控制器中基于特征的 DDoS 检测和缓解方案,并设计了一种基于距离的分类方法来检测合法数据包中的恶意数据包。Sanmorino 等人在应对 DDoS 攻击时,探讨了流入口模式检测方法及分层防火墙,测试了网络处于正常、不安全和安全状态下的三种场景,结果表明该算法不仅能缓解 DDoS 攻击,还能确保主机始终正常运行。

Ye 等通过统计流表中的信息并手动构造特征,然后利用支持向量机算法进行分类,可以有效检测当前的 DDoS 攻击,并预测尚未被发现的攻击。
\subsection{SDN 安全研究现状}
SDN 安全研究主要分为两大流派:基于统计的方法和基于机器学习的方法。

基于统计的检测方案:这种方法通过统一提取攻击流量的统计特征来区分流量。通常会设置一个上限,超过该上限的流量会被识别为攻击流量。Rui Wang 等人设计了一种方法,在网络中关键节点的交换机端增加数据收集器进行测算,将目的IP地址相同的流量归为一类,计算其熵值,并提出了一种基于此值的检测方法。该方法高效且快速,但其缺点是额外的收集器对网络系统有一定的侵入性。Mousavi 等人在上述算法的基础上,将熵值测算周期改为每50个数据包计算一次,降低了实时测算时系统的负担,但在实际环境中效果一般,当入侵流量占比较低时,容易误判,导致严重损失。Shariq 等人通过评估和判断流表项中的数据包,包括源IP是否在流表中以及是否完成 TCP 连接,通过动态自适应地改变阈值来确定网络是否受到攻击。

基于机器学习的检测方案:这种方法通过对流经交换机的流量数据进行建模和特征提取,然后利用机器学习算法识别恶意流量。常用的方法包括SOM、SVM和神经网络。Braga 等人使用轻量级的SOM来判断攻击的发生。Ye 等通过 Onp Flow Stats 信息收集流表的统计信息,构造多个数据特征,最后使用SVM算法进行入侵流量的判定。类似地,Mehr 等人也采用了相同的分类算法,但在实验过程中未能证明该检测方法的可信度。
\subsection{传统网络中的入侵检测技术}
入侵检测技术是网络安全防护中不可或缺的重要组成部分,通过选择异常检测或误用检测的方式,为预防入侵行为提供了一种高效的应对方法,从而发现未授权或具有威胁的系统及网络行为。在网络信息高度发达的今天,网络攻击事件的发生频率和规模逐年上升,使得入侵检测成为网络安全领域的研究热点。

随着网络环境的日益复杂和多样化,入侵检测面临新的挑战,例如海量数据难以处理、高维数据特征提取困难以及新型攻击手段的不断涌现。深度学习在处理高维度、高复杂度的网络流量数据时具有较强的特征提取能力,通过关注输入数据间的表征提取来提高模型的分类精度。因此,基于深度学习的入侵检测技术已成为当前研究的关键。深度学习方法根据是否使用带标签的数据,可以分为三类:有监督的深度学习、无监督的深度学习和混合式的深度学习。
\section{论文的主要研究内容}
%\subsubsection{研究目标}
%\paragraph{研究目标1(4级节标题)}
%\subparagraph{研究目标1.1(5级节标题)}

\begin{enumerate}
    \item [*]章标题,单倍行距,段前24磅,段后18磅
    \item [*]一级标题,单倍行距,段前24磅,段后6磅
    \item [*]二级标题,单倍行距,段前12磅,段后6磅
    \item [*]三级标题,单位行距,段前12磅,段后6磅
    \item [*]四到刘级标题,单位行距,段前段后均为0磅。
\end{enumerate}

\section{论文组织结构}