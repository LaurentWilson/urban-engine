% !TeX root = ./main.tex
% 首先填写论文题目等个人信息。
\custsetup{
  title              = {应对 DDoS 攻击的 SDN 入侵检测算法研究},
  title*             = {Research on SDN Intrusion Detection Algorithm for Mitigating DDoS Attacks},
  author             = {梁浩华},%作者中文名
  author*            = {Liang Haohua},%作者英文名
  speciality         = {通信工程},%学科专业或类别领域
  speciality*        = {Information Warfare Technology},%学科专业或类别领域英文
  supervisor         = {张竟秋},%指导教师中文
  supervisor*        = {Zhang jingqiu},%指导教师英文
  %advisor            = {XXX研究院AAA~研究员},%专业学位的企业导师拼写,如没有,请忽略或注释掉
%  advisor*           = {Prof. AAA},%专业学位企业导师英文拼写,《研究生规范中》暂未要求企业导师英文拼写内容。
  degree-area*      = {Master of Engineering},%硕士类型英文,工学硕士,理学说是为Master of Science,博士学位为Doctor of Philosophy
  % date               = {2017-05-01},  % 默认为今日
  start-date = {2017年11月}, %论文初始时间
  department         = {电子信息工程学院},  % 院系,本科生需要填写
  student-id         = {200421205},  % 学号
  % 如果论文不涉密,则将下面secret-level,secret-level*和secret-year注释掉。
  secret-level       = {秘密},     % 绝密|机密|秘密|控阅,注释本行则论文为非涉密论文,可以公开
  secret-level*      = {Secret},  % Top secret | Highly secret | Secret
  secret-year        = {10},      % 保密/控阅期限
  %
  % 数学字体
  % math-style         = GB,  % 可选:GB, TeX, ISO
  math-font          = xits,  % 可选:stix, xits, libertinus
}
% 加载宏包

% 定理类环境宏包
\usepackage{amsthm}

% 插图
\usepackage{graphicx}

% 三线表
\usepackage{booktabs}

% 跨页表格
\usepackage{longtable}

% 算法
\usepackage[ruled,linesnumbered]{algorithm2e}

% SI 量和单位
\usepackage{siunitx}

% 参考文献使用 BibTeX + natbib 宏包
% 顺序编码制
\usepackage[sort]{natbib}
\bibliographystyle{custhesis-numerical}

% 著者-出版年制
% \usepackage{natbib}
% \bibliographystyle{custhesis-authoryear}

% 本科生参考文献的著录格式
% \usepackage[sort]{natbib}
% \bibliographystyle{custhesis-bachelor}

% 参考文献使用 BibLaTeX 宏包
% \usepackage[style=custhesis-numeric]{biblatex}
% \usepackage[bibstyle=custhesis-numeric,citestyle=custhesis-inline]{biblatex}
% \usepackage[style=custhesis-authoryear]{biblatex}
% \usepackage[style=custhesis-bachelor]{biblatex}
% 声明 BibLaTeX 的数据库
% \addbibresource{bib/cust.bib}

% 配置图片的默认目录
\graphicspath{{figures/}}

% 数学命令
\makeatletter
\newcommand\dif{%  % 微分符号
  \mathop{}\!%
  \ifcust@math@style@TeX
    d%
  \else
    \mathrm{d}%
  \fi
}
\makeatother
\newcommand\eu{{\symup{e}}}
\newcommand\iu{{\symup{i}}}

% 用于写文档的命令
\DeclareRobustCommand\cs[1]{\texttt{\char`\\#1}}
\DeclareRobustCommand\env{\texttt}
\DeclareRobustCommand\pkg{\textsf}
\DeclareRobustCommand\file{\nolinkurl}

% hyperref 宏包在最后调用
\usepackage{hyperref}
